%%
%% This is file `sample-sigconf.tex',
%% generated with the docstrip utility.
%%
%% The original source files were:
%%
%% samples.dtx  (with options: `sigconf')
%% 
%% IMPORTANT NOTICE:
%% 
%% For the copyright see the source file.
%% 
%% Any modified versions of this file must be renamed
%% with new filenames distinct from sample-sigconf.tex.
%% 
%% For distribution of the original source see the terms
%% for copying and modification in the file samples.dtx.
%% 
%% This generated file may be distributed as long as the
%% original source files, as listed above, are part of the
%% same distribution. (The sources need not necessarily be
%% in the same archive or directory.)
%%
%% The first command in your LaTeX source must be the \documentclass command.
\documentclass[$format$, $if(screen)$screen,$endif$$if(review)$review,$endif$$if(anonymous)$anonymous,$endif$$if(authorversion)$authorversion,$endif$]{acmart}

\def\tightlist{} % UL 21 Aug 2020 stop pandoc from messing things up by putting lists in 'tightlist' command when no space between bullet points in Rmd file

%%
%% \BibTeX command to typeset BibTeX logo in the docs
\AtBeginDocument{%
  \providecommand\BibTeX{{%
    \normalfont B\kern-0.5em{\scshape i\kern-0.25em b}\kern-0.8em\TeX}}}

% UL 20 Aug 2020, enable code inclusion
$if(highlighting-macros)$
$highlighting-macros$
$endif$

%% Rights management information.  This information is sent to you
%% when you complete the rights form.  These commands have SAMPLE
%% values in them; it is your responsibility as an author to replace
%% the commands and values with those provided to you when you
%% complete the rights form.
\setcopyright{$setcopyright$}
\copyrightyear{$copyright-year$}
\acmYear{$acm-year$}
\acmDOI{$acmDOI$}

%% These commands are for a PROCEEDINGS abstract or paper.
\acmConference[$conference-short$]{$conference$}{$conference-date$}{$conference-location$}
\acmBooktitle{$acm-booktitle$}
\acmPrice{$acm-price$}
\acmISBN{$acmISBN$}


%%
%% Submission ID.
%% Use this when submitting an article to a sponsored event. You'll
%% receive a unique submission ID from the organizers
%% of the event, and this ID should be used as the parameter to this command.
%%\acmSubmissionID{123-A56-BU3}

%%
%% The majority of ACM publications use numbered citations and
%% references.  The command \citestyle{authoryear} switches to the
%% "author year" style.
%%
%% If you are preparing content for an event
%% sponsored by ACM SIGGRAPH, you must use the "author year" style of
%% citations and references.
%% Uncommenting
%% the next command will enable that style.
%%\citestyle{acmauthoryear}

%%
%% end of the preamble, start of the body of the document source.
\begin{document}

%%
%% The "title" command has an optional parameter,
%% allowing the author to define a "short title" to be used in page headers.
\title[$short-title$]{$title$}

%%
%% The "author" command and its associated commands are used to define
%% the authors and their affiliations.
%% Of note is the shared affiliation of the first two authors, and the
%% "authornote" and "authornotemark" commands
%% used to denote shared contribution to the research.
$for(author)$
\author{$author.name$}
$if(author.authornote)$
\authornote{$author.authornote$}
$endif$
$if(author.authornotemark)$
\authornotemark[$author.authornotemark$]
$endif$
$if(author.affiliation)$
\affiliation{
  \institution{$author.affiliation.institution$}
  \streetaddress{$author.streetaddress$}
  \city{$author.affiliation.city$}
  \state{$author.affiliation.state$}
  \country{$author.affiliation.country$}
  \postcode{$author.affiliation.postcode$}
}
$endif$
$if(author.additional-affiliation)$
\additionalaffiliation{
  \institution{$author.additional-affiliation.institution$}
  \city{$author.additional-affiliation.city$}
  \state{$author.additional-affiliation.state$}
  \country{$author.additional-affiliation.country$}
  \postcode{$author.additional-affiliation.postcode$}
}
$endif$
\email{$author.email$}
$if(orcid)$
  \orcid{$orcid$}
$endif$
$endfor$

%%
%% By default, the full list of authors will be used in the page
%% headers. Often, this list is too long, and will overlap
%% other information printed in the page headers. This command allows
%% the author to define a more concise list
%% of authors' names for this purpose.
\renewcommand{\shortauthors}{$short-authors$}

%%
%% The abstract is a short summary of the work to be presented in the
%% article.
\begin{abstract}
$abstract$
\end{abstract}

% UL 20 Aug 2020 the ccsxml.tex will be inserted here
$for(header-includes)$
$header-includes$
$endfor$


%%
%% Keywords. The author(s) should pick words that accurately describe
%% the work being presented. Separate the keywords with commas.
\keywords{$keywords$}

%% A "teaser" image appears between the author and affiliation
%% information and the body of the document, and typically spans the
%% page.
$if(teaser-figure)$
  \begin{teaserfigure}
    \includegraphics[width=\textwidth]{$teaser-figure$}
    \caption{$teaser-caption$}
    \Description{$teaser-description$}
  \label{fig:$teaser-label$}
\end{teaserfigure}
$endif$

%%
%% This command processes the author and affiliation and title
%% information and builds the first part of the formatted document.
\maketitle

% insert text from main.Rmd here
$body$

%%
%% The next two lines define the bibliography style to be used, and
%% the bibliography file.
\bibliographystyle{ACM-Reference-Format}
\bibliography{$bibliography$}

\end{document}
\endinput
%%
%% End of file `sample-sigconf.tex'.
